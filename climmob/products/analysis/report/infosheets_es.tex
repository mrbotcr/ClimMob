
\documentclass[10pt]{article}
\usepackage{graphicx}
\usepackage[bottom=0.5in]{geometry}
\usepackage[utf8]{inputenc}
\usepackage{tabularx}
\usepackage{longtable}

\begin{document}
\begin{titlepage}
\pagenumbering{gobble}
\setlength{\voffset}{-0.5in}
\setlength{\parindent}{1em}
\setlength{\parskip}{1em}
\renewcommand{\baselinestretch}{1.5}
\rmfamily




	\textbf{Gracias por su participación!}
	\newline
	\newline
	{{ u.uname }} \newline
	{{ u.family }} \newline
	{{ u.div2 }} \newline
	{{ u.div1 }} \newline


	Estos son los resultados del experimento al que contribuiste.

	\begin{flushleft}
		\textbf{Recibiste las siguientes variedades para clasificar: }\hfill \break
		%------------------
		\begin{tabularx}{\textwidth}{ X | c  }
			\hline
			\textbf{Variedad} & \textbf{Nombre} \\ \hline

			
				{{v.var}} & {{v.name}} \\ \hline
			


		\end{tabularx}\newline \newline

		\textbf{Clasificó estas variedades en el siguiente orden: }\hfill \break
		\begin{tabularx}{\textwidth}{ X | c | c | c  }
			\hline
			\textbf{Características}
			
				& \textbf{ {{loop.index}} }
			
			\\ \hline

			%\textbf{Characteristic} & \textbf{Best} & \textbf{Second} & \textbf{Worst} \\ \hline

			
				%{{c.carac}} & {{c.pos}} & {{c.second}} & {{c.worst}} \\ \hline
				{{c.carac}} & {{r}}  \\ \hline


			

		\end{tabularx}

	\end{flushleft}

	\pagebreak

	\begin{flushleft}
		Estas son las mejores y peores variedades que usted y los observadores con las características similares recibieron:\hfill \break \newline
		%------------------
		\begin{tabularx}{\textwidth}{ X | X  }
			\hline
			\textbf{Posición} & \textbf{Nombre} \\ \hline

			

				{{p.pos}} & {{p.var}} \\ \hline
			



		\end{tabularx}\newline \newline

 \begin{longtable}{|*3{p{2cm}|}}
    \hline
    {\bf Primera} & {\bf Segunda} & {\bf Tercera} \\ \hline

    Text   & Other Text    & Other Text 2 \\
           & Other Text 3  &              \\
           & Other Text 4  &              \\ \hline

    Text 5 & Other Text 6  & Other Text 7 \\
           & Other Text 8  &              \\
           & Other Text 9  &              \\ \hline
\end{longtable}

\begin{center}
  \begin{tabular}{ l | c | r }
    \hline
    1 & 2 & 3 \\ \hline
    4 & 5 & 6 \\ \hline
    7 & 8 & 9 \\
    \hline
  \end{tabular}
\end{center}



	\end{flushleft}
	\pagebreak




\end{titlepage}

\end{document}	


